\section{Нормальная стратегия редукций, теорема об успешности нормальной стратегии.}

\begin{definition}[Один редекс левее другого]
    когда $\lambda$ первого редекса расположена левее (в записи терма) $\lambda$ второго редекса.
\end{definition}

Это означает, что либо первый редекс целиком
расположен левее второго, либо второй редекс находится
внутри первого.

\begin{definition}[Нормальная стратегия редукций]
    всегда редуцируй самый левый редекс.
\end{definition}

При этом это не обязательно самая левая $\lambda$ - левее могут быть лямбды, не образующие $\beta$-редексов.

\begin{theorem}[Успешность нормальной стратегии.]
    Если терм можно привести к нормальной форме, то нормальная стратегия добьётся этого.
\end{theorem}
