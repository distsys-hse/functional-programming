\section{Исчисление типизации \texorpdfstring{$\lambda_{\to}$}{} по Карри.}

%Типы собираются из \textit{базовых типов}, которые делятся на \textit{переменные} ($r_1, r_2, \ldots$) и \textit{константы} ($p_1, p_2, \ldots$).

\begin{definition}[Стрельчатый тип]
функциональный тип ($A \to B$).
\end{definition}


Типы собираются из переменных по типам ($r_1, r_2, \ldots$) и констант по типам ($p_1, \ldots$) только с помощью операции $\to$ ($A, B$ --- типы, то $A \to B$ --- тип).

Ограничение только на применение: $(vu)$ корректно $\leftrightarrow$ $u$ имеет тип $A$, а $v$ имеет тип $A \to B$.

$\lambda$-абстракция может применяться всегда, образуя стрельчатый тип.

Типы переменных указываются в контексте $\Gamma = x_1 : A_1, x_2: A_2, \ldots$.

\begin{definition}[Типизация по Карри]
полиморфная система типов.
Терм в контексте может иметь много различных типов.
\end{definition}

\begin{important}
Типизация термов со свободными переменными осуществляется в контексте.
\end{important}

\begin{definition}[Утверждение о типизуемости]
запись вида $\Gamma \vdash u : B$, которая обозначает, что $B$ -- допустимый тип для $u$ в контексте $\Gamma$.

Эти утверждения могут доказываться.
\end{definition}

Терм $u$ \textbf{не типизируем в $\Gamma$}, если не доказуемо $\Gamma \vdash u : B$ ни для какого $B$.

Для доказательства типизируемости используются следующие правила исчисления типизации по Карри.

\[
\frac{}{\Gamma, x:A \vdash x:A} Ax \text{ аксиома}
\]

\[
\frac{\Gamma, x:A \vdash u:B}{\Gamma \vdash (\lambda x.u) : (A \to B)} Abs
\]

\[
\frac{\Gamma \vdash u : (A \to B) \hspace{0.7cm} \Gamma \vdash v : A}{\Gamma \vdash (uv):B} App
\]


То что ниже, наверное, можно и не рассказывать.

\begin{theorem}[О нормализуемости]
Любой типизируемый терм сильно нормализуем.
\end{theorem}

\begin{corollary}
Комбинатор неподвижной точки $\Y$ в $\lambda_{\to}$ не типизируем. Это можно исправить, если ввести константу $\mathbb{Y}$ с полиморфным типом $(r \to r) \to r$ и $\sigma$-редукцию $\mathbb{Y} u \to_\sigma u(\mathbb{Y} u)$.

Однако возникнет проблема, что его потребуется включить в контекст, где $r$ зафиксируется. 
\end{corollary}


