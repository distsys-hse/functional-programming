\section{Нормальная форма. Сильная и слабая нормализуемость. Примеры термов с разными свойствами нормализуемости.}


\begin{definition}[Нормальная форма]
терм без $\beta$-редексов (нельзя далее редуцировать).
\end{definition}

\begin{corollary}
Из-за свойства Чёрча-Россера, если НФ существует, то она единственна.
\end{corollary}

\begin{definition}[Сильно нормализуемые термы]
термы, которые приводимы к нормальной форме при любой последовательности (пути) редукций.
\end{definition}

\begin{definition}[Слабо нормализуемые термы]
термы, которые приводимы к нормальной форме при одной последовательности (пути) редукций, а при другой редукции не завершаются (путь бесконечен).
\end{definition}

\begin{example}
$\omega = \lambda x.(xx)$, $\Omega = \omega\omega$.
Терм $\Omega$ сводится редукциями к себе, следовательно он не нормализуем.

\[
\Omega = \omega\omega =  (\lambda x.(xx))(\lambda x.(xx)) \rightarrow_{\beta} (\lambda x.(xx))(\lambda x.(xx)) = \omega\omega = \Omega
\]
\end{example}

\begin{example}
$(\lambda x. y)\Omega$ является слабо нормализуемым.
Если редуцируем сначала $\Omega$, то попадаем в бесконечный цикл, иначе сразу получаем НФ.
\end{example}

\begin{example}
$(\lambda x.x)y$ является сильно нормализуемым
\end{example}
