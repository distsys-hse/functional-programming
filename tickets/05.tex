\section{Комбинаторы неподвижной точки, пример: комбинатор \texorpdfstring{$\Y$}{}.}

Для достижения полноты по Тьюрингу требуется рекурсия.
Мы не хотим, чтобы функция определялась через саму себя, поэтому требуется комбинатор неподвижной точки.

\begin{definition}[Неподвижная точка функции $f$]
такой терм $F$, что для функции $f$ выполнено $F =_\beta f F$.
\end{definition}

Символ $=_\beta$ означает $\beta$-эквивалентность – $a =_\beta b$, если $\exists c : a \to_\beta c \land b \to_\beta c$

\begin{definition}[Комбинатор неподвижной точки]
функция высшего порядка, вычисляющая неподвижную точку другой функции.
\end{definition}

\begin{example}[$\Y$-комбинатор]
Основное свойство $\Y f =_\beta f(\Y f)$.
\begin{align}
    \Y = \lambda f.\left((\lambda x. f(xx)) (\lambda x. f(xx))\right)
\end{align}
\end{example}

\begin{important} 
Функция с $\Y$ комбинатором не может быть сильно нормализуемой, поэтому важна нормальная стратегия из вопроса 4.
\end{important}

\begin{example} 
Факториал можно выразить следующим образом:
\begin{align}
    \operatorname{Fact} = \Y \left( \lambda g . \lambda x . (
        \text{if Zero } x 
        \text{ then } 1 
        \text{ else } (g(\text{Prev } x) \cdot x)\right)
    ),
\end{align}
где Zero -- проверка на $0$, а Prev -- предыдущее число (тут используются нумералы Чёрча).
\end{example}
