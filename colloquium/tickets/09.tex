\section{Безопасность типов при \texorpdfstring{$\beta$}{}-редукции.}

\begin{definition}[Свойство безопасности типов при $\beta$-редукции]
если $\Gamma \vdash u : B$ и $u\twoheadrightarrow_\beta u'$, то $\Gamma \vdash u' : B$
\end{definition}
Это означает, что в процессе вычислений можно не
контролировать типы, достаточно (статической) проверки
в начале.
\begin{important}
В обратную сторону условие не гарантируется: если $u\twoheadrightarrow_\beta u'$ и $\Gamma \vdash u' : B$, не обязательно $\Gamma \vdash u : B$. Может оказаться,
что $u$ вообще не типизируем, либо у него меньше
корректных типов, чем у $u'$.
\end{important}

\begin{example} 
Возможно всё-таки стоит привести пример когда обратное условие не выполняется.
Если кто-то придумает такой пример - напишите его сюда пожалуйста.
\end{example}

\textcolor{red}{Думаю, что подойдет $(\lambda x. y)\Omega$ про не типизируемость и $(\lambda f t. (\lambda x y. y) f(42) t ))$ для меньшего числа корректных типов, т.к. после редукции внутреннеко редекса $f$ может быть любым термом, а не только $Int -> smth$}